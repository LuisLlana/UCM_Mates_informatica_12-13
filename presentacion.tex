% $Header: /cvsroot/latex-beamer/latex-beamer/solutions/conference-talks/conference-ornate-20min.en.tex,v 1.7 2007/01/28 20:48:23 tantau Exp $

\documentclass[handout]{beamer}
%\documentclass{beamer}
% This file is a solution template for:

% - Talk at a conference/colloquium.
% - Talk length is about 20min.
% - Style is ornate.



% Copyright 2004 by Till Tantau <tantau@users.sourceforge.net>.
%
% In principle, this file can be redistributed and/or modified under
% the terms of the GNU Public License, version 2.
%
% However, this file is supposed to be a template to be modified
% for your own needs. For this reason, if you use this file as a
% template and not specifically distribute it as part of a another
% package/program, I grant the extra permission to freely copy and
% modify this file as you see fit and even to delete this copyright
% notice. 


\mode<presentation>
{
  %\usetheme{Berlin}%{Warsaw}
  %\usetheme{Boadilla}%{Warsaw}
  \usetheme{Madrid}%{Warsaw}
  % or ...

  \setbeamercovered{transparent}
  % or whatever (possibly just delete it)
  
  %\setbeameroption{show notes}


}


%\setbeameroption{show notes on second screen=left}%{show notes}

%\usepackage[english]{babel}
\usepackage[spanish]{babel}
% or whatever
\usepackage[latin9]{inputenc}
% or whatever

\usepackage{times}
\usepackage[T1]{fontenc}
% Or whatever. Note that the encoding and the font should match. If T1
% does not look nice, try deleting the line with the fontenc.




\title[Inform\'atica] % (optional, use only with long paper titles)
{Inform\'atica}
\subtitle{Grado en Matem�ticas \\ Grado en Ingenier�a Matem�tica \\  Grado en Matem�ticas y Estad�stica \\
Grupo D}
\author[Escribano-Llana]{Jes�s Escribano Mart�nez\\Luis Llana D�az}
\institute[UCM]{Departamento de Sistemas Inform�ticos y Computaci�n\\Universidad Complutense de Madrid}
\date{Curso 2012/2013}

% - Give the names in the same order as the appear in the paper.
% - Use the \inst{?} command only if the authors have different
%   affiliation.

% - Use the \inst command only if there are several affiliations.
% - Keep it simple, no one is interested in your street address.


% - Either use conference name or its abbreviation.
% - Not really informative to the audience, more for people (including
%   yourself) who are reading the slides online

\subject{Theoretical Computer Science}
% This is only inserted into the PDF information catalog. Can be left
% out. 



% If you have a file called "university-logo-filename.xxx", where xxx
% is a graphic format that can be processed by latex or pdflatex,
% resp., then you can add a logo as follows:

% \pgfdeclareimage[height=0.5cm]{university-logo}{university-logo-filename}
% \logo{\pgfuseimage{university-logo}}



% Delete this, if you do not want the table of contents to pop up at
% the beginning of each subsection:
% \AtBeginSubsection[]
% {
%   \begin{frame}<beamer> \frametitle{Outline}
%     \tableofcontents[currentsection,currentsubsection]
%   \end{frame}
% }


% If you wish to uncover everything in a step-wise fashion, uncomment
% the following command: 
%\beamerdefaultoverlayspecification{<+->}


\begin{document}

\begin{frame}
  \titlepage
\end{frame}

% \begin{frame}
% \frametitle{Outline}
%   \tableofcontents
%   % You might wish to add the option [pausesections]
% \end{frame}


\section{Presentacion}

\subsection{El profesor}

\begin{frame}
\frametitle{El profesor}
%\framesubtitle{Subtitles are optional.}
 

  \begin{itemize}
  \item \textbf{Nombre}  Jes\'us Escribano Mart\'inez
 \item \textbf{Departamento}  Sistemas Inform\'aticos y Computaci\'on
 \item \textbf{Despacho} 452, Facultad de CC. Matem\'aticas
 \item \textbf{Telef.}  91 394 4679
\item \textbf{email} \texttt{jesus\underline{ }escribano@mat.ucm.es}
\item \textbf{Tutorias Oficiales}  1$^{er}$C: L,M: 11:00-13:00. 2$^o$C:  L: 11:00-13:00, J: 10:00-12:00
  \end{itemize}

\pause
Para las tutorias, es mejor concertar una \alert{cita previa} con 
antelaci\'on.
\end{frame}


\begin{frame}
\frametitle{Otros profesores}
%\framesubtitle{Subtitles are optional.}
 
\begin{itemize}
\item Luis Llana (2$^o$ C, Teor�a)
\item Antonio Sarasa (1$^{er}$ C, pr�cticas)
\item Ignacio Casti�eiras (1$^{er}$ C, pr�cticas)
\item Adri�n Riesco (2$^o$ C, pr�cticas)
\end{itemize}


\end{frame}


\begin{frame}
\frametitle{Comunicaci\'on}

La comunicaci\'on del curso se realizar\'a 
a trav\'es del Campus Virtual  \\
\texttt{http://www.ucm.es/campusvirtual/}\pause


Debes acceder habitualmente a dos espacios
\begin{itemize}
\item \alert{Inform�tica FCM, 2012-13, grupo com�n} donde est� toda la 
informaci�n de la asignatura, com�n a todos los grupos. 
\item \alert{INFORMATICA (1� Grupo D, 2012/2013)} donde est� toda la 
informaci�n espec�fica del grupo D.
\end{itemize}

\end{frame}

\begin{frame}
\frametitle{Comunicaci\'on}

\begin{itemize}
\item \alert{TODA LA INFORMACI�N DEL CURSO ESTAR� EN EL CAMPUS VIRTUAL} No olvides
revisarlo con regularidad. \pause
\item Comprueba que los datos son correctos.
  \begin{itemize}
  \item Inserta una foto tuya reciente.
    \item  A�ade una direcci�n de e-mail que utilices normalmente, ser� una v�a directa de comunicaci�n.
  \end{itemize}
\item Se pondr\'a todo tipo de informaci\'on sobre el curso \pause
  \begin{itemize}
  \item Materiales del curso       \pause
  \item  Citas para tutorias       \pause
  \item Citas para entrega de pr\'acticas
  \end{itemize}
\end{itemize}

\end{frame}


\subsection{Contenido del curso}


\begin{frame}
\frametitle{Contenido del curso}
%\framesubtitle{The proof uses \textit{reductio ad absurdum}.}

Este curso consiste en una \alert{Introducci\'on a la Programaci\'on}.\pause

Los conceptos b\'asicos son \emph{independientes} del lenguaje de programaci\'on utilizado. \pause

Utilizaremos un lenguaje concreto: \alert{python}.\\
\texttt{http://www.python.org/} \pause

Texto b�sico: 
Andr�s Marzal e Isabel Garc�a 
\alert{Introducci�n a la Programaci�n con Python} \\
Accesible en \texttt{https://arco.esi.uclm.es/public/doc/book/python.pdf}\pause

Hay todo tipo de informaci�n adicional en el campus virtual.


\end{frame}


\begin{frame}
\frametitle{Evaluaci\'on}
\begin{block}{Ex\'amenes finales``tradicionales''}
Convocatorias de  Junio y Septiembre. 
\alert{50 \% de la nota.}
\end{block} \pause
\begin{block}{Pr\'actica obligatorias}
La realizaci�n de una pr\'actica consiste en la escritura de un programa que resuelva un determinado problema. El c\'odigo del programa ser� entregado al profesor y se defender� en una entrevista personal. 
\alert{40 \% de la nota.}
\end{block}
\pause

\begin{block}{Asistencia y Participaci�n}
Se valorar� la asistencia regular a las clases, y la participaci�n en las mismas. 
\alert{10\% de la nota.}
\end{block}
\pause
Para valorar una nota en los apartados anteriores, 
es necesario que esta nota alcance al menos un 40\% de la nota m�xima posible 
en el apartado correspondiente. 

%Ya hay \alert{hoja de ejercicios} disponibles en el Campus Virtual. 

\end{frame}

\begin{frame}
\begin{center}
\LARGE{EMPEZAMOS \dots }
\end{center}
\end{frame}


\end{document}


%%%%%%%%%%%%%%%%%%%%%%
%%%%%%%%%%%%%%%%%%%%%%


\begin{frame}
\frametitle{What's Still To Do?}
\begin{columns}
\column{.5\textwidth}
\begin{block}{Answered Questions}
How many primes are there?
\end{block} \pause
\column{.5\textwidth}
\begin{block}{Open Questions}
Is every even number the sum of two primes?
\end{block}
\end{columns}
\end{frame}



\begin{frame}[fragile]
\frametitle{An Algorithm For Finding Primes Numbers.}
\begin{verbatim}
program probando;

begin
   writeln('Qu� bonito es el beamer');
end.

\end{verbatim}
\begin{uncoverenv}<2>
Nice algorithm.
\end{uncoverenv}

\end{frame}







\begin{frame}
\frametitle{$P = NP?$}
\begin{theorem}
Se verifica que $P = NP$.
\end{theorem}
\begin{proof}
?`No es evidente?
\end{proof}
\end{frame}



\begin{frame}
\frametitle{Resaltando}
Aunque parezca mentira, \structure{esto es muy importante}.
Adem\'as, esto es todav�a \structure{m�s importante}.
\begin{itemize}
\item<1-> First point, shown on all slides.
\item<2-> Second point, shown on slide 2 and later.
\item<2-> Third point, also shown on slide 2 and later.
\item<3-> Fourth point, shown on slide 3.
\end{itemize}
\end{frame}
\begin{frame}
\begin{enumerate}
\item<3-| alert@3>[0.] A zeroth point, shown at the very end.
\item<1-| alert@1> The first and main point.
\item<2-| alert@2> The second point.
\end{enumerate}
\end{frame}




\section{Our Results/Contribution}

\subsection{Main Results}

\begin{frame}
\frametitle{Es �til trabajar con bloques}
\begin{block}{primer bloque}
  \begin{enumerate}
  \item Primer bloque
 \item Primer bloque
  \end{enumerate}
\end{block}

\pause

\begin{block}{segundo bloque}
  \begin{itemize}
  \item segundo bloque
  \item segundo bloque
  \item segundo bloque
  \end{itemize}
\end{block}
\end{frame}

\begin{frame}
\frametitle{Cajas de colores}

\setbeamercolor{rojo}{fg=black,bg=red}
\begin{beamercolorbox}{rojo}
Podemos poner cajas de colores
\end{beamercolorbox}
\pause
\setbeamercolor{azul}{fg=black,bg=blue}
\begin{beamercolorbox}{azul}
Podemos poner cajas de colores, y luego cambiar el color.
\end{beamercolorbox}
\pause

\setbeamercolor{postit}{fg=black,bg=yellow}
\begin{beamercolorbox}[sep=1em,wd=5cm]{postit}
Place me somewhere!
\end{beamercolorbox}


\end{frame}

\begin{frame}
\frametitle{Uso de columnas}

\begin{columns}[t]
\begin{column}{5cm}
En la primera columna\\
ponemos varias lineas. 
\end{column}
\pause
\begin{column}{5cm}
En la segundo columna ponemoso solo una 
linea, pero alineada.  
\end{column}
\end{columns}

\end{frame}


\subsection{Basic Ideas for Proofs/Implementation}

\begin{frame}
\frametitle{Make Titles Informative.}
\end{frame}

\begin{frame}
\frametitle{Make Titles Informative.}
\end{frame}

\begin{frame}
\frametitle{Make Titles Informative.}
\end{frame}



\section*{Summary}

\begin{frame}
\frametitle{Summary}

  % Keep the summary *very short*.
  \begin{itemize}
  \item
    The \alert{first main message} of your talk in one or two lines.
  \item
    The \alert{second main message} of your talk in one or two lines.
  \item
    Perhaps a \alert{third message}, but not more than that.
  \end{itemize}
  
  % The following outlook is optional.
  \vskip0pt plus.5fill
  \begin{itemize}
  \item
    Outlook
    \begin{itemize}
    \item
      Something you haven't solved.
    \item
      Something else you haven't solved.
    \end{itemize}
  \end{itemize}
\end{frame}



% All of the following is optional and typically not needed. 
\appendix
\section<presentation>*{\appendixname}
\subsection<presentation>*{For Further Reading}

\begin{frame}[allowframebreaks]
  \frametitle<presentation>{For Further Reading}
    
  \begin{thebibliography}{10}
    
  \beamertemplatebookbibitems
  % Start with overview books.

  \bibitem{Author1990}
    A.~Author.
    \newblock {\em Handbook of Everything}.
    \newblock Some Press, 1990.
 
    
  \beamertemplatearticlebibitems
  % Followed by interesting articles. Keep the list short. 

  \bibitem{Someone2000}
    S.~Someone.
    \newblock On this and that.
    \newblock {\em Journal of This and That}, 2(1):50--100,
    2000.
  \end{thebibliography}
\end{frame}

\end{document}




